%  LaTeX support: latex@i-edu.uz
%  For support, please attach all files needed for compiling as well as the log file, and specify your operating system, LaTeX version, and LaTeX editor.

%=================================================================
\documentclass[i-edu.uz,journal,article,submit,pdftex,moreauthors]{Definitions/i-edu.uz} 
\usepackage[T2A]{fontenc}
\usepackage[utf8]{inputenc} % If you're using UTF-8 encoding
\usepackage[russian]{babel}
\usepackage{tempora} % Or any other suitable font package

%---------
% article
%---------
% The default type of manuscript is "article", but can be replaced by: 
% аннотации, приложение, статья, книга, рецензия на книгу, краткий отчет, отчет о случае, комментарий, комментарий, сообщение, материалы конференции, исправление, отчет о конференции, запись, выражение озабоченности, расширенный реферат, дескриптор данных, редакционная статья, эссе, опечатка, гипотеза, интересное изображение, некролог, мнение, отчет о проекте , ответ, опровержение, обзор, перспектива, протокол, короткая заметка, протокол исследования, систематический обзор, файл поддержки, техническая заметка, точка зрения, рекомендации, зарегистрированный отчет, руководство
% supfile = supplementary materials

%----------
% submit
%----------
% Опция класса «отправить» будет изменена редакцией на «принять», когда статья будет принята. Это приведет к изменению только главной страницы (например, станет виден логотип журнала), заголовков и информации об авторских правах. Также будет удалена нумерация строк. Информация о журнале и нумерация страниц для принятых статей также определяются редакцией.

%------------------
% moreauthors
%------------------
% Если имеется только один автор, следует использовать опцию класса oneauthor. В противном случае используйте опцию класса moreauthors.

%---------
% pdftex
%---------
% Опция pdftex предназначена для использования с pdfLaTeX. Удалите «pdftex» для (1) компиляции с помощью LaTeX и dvi2pdf (если используются рисунки в формате EPS) или для (2) компиляции с XeLaTeX.

%=================================================================
% i-edu.uz internal commands - do not modify
\firstpage{1} 
\makeatletter 
\setcounter{page}{\@firstpage} 
\makeatother
\pubvolume{4}
\issuenum{1}
\articlenumber{2}
\pubyear{2024}
\copyrightyear{2024}
%\externaleditor{Academic Editor: Firstname Lastname}
\datereceived{10.01.2024 } 
\daterevised{18.01.2024} % Comment out if no revised date
\dateaccepted{25.03.2024} 
\datepublished{30.03.2024} 
%\datecorrected{} % For corrected papers: "Corrected: XXX" date in the original paper.
%\dateretracted{} % For corrected papers: "Retracted: XXX" date in the original paper.
\hreflink{https://doi.org/10.70626/cardiouz-2024-4-00000} % If needed use \linebreak
%\doinum{}
%\pdfoutput=1 % Uncommented for upload to arXiv.org
%\CorrStatement{yes}  % For updates


%=================================================================
% Add packages and commands here. The following packages are loaded in our class file: fontenc, inputenc, calc, indentfirst, fancyhdr, graphicx, epstopdf, lastpage, ifthen, float, amsmath, amssymb, lineno, setspace, enumitem, mathpazo, booktabs, titlesec, etoolbox, tabto, xcolor, colortbl, soul, multirow, microtype, tikz, totcount, changepage, attrib, upgreek, array, tabularx, pbox, ragged2e, tocloft, marginnote, marginfix, enotez, amsthm, natbib, hyperref, cleveref, scrextend, url, geometry, newfloat, caption, draftwatermark, seqsplit, russian
% cleveref: load \crefname definitions after \begin{document}

%=================================================================
% Please use the following mathematics environments: Theorem, Lemma, Corollary, Proposition, Characterization, Property, Problem, Example, ExamplesandDefinitions, Hypothesis, Remark, Definition, Notation, Assumption
%% For proofs, please use the proof environment (the amsthm package is loaded by the i-edu.uz class).

%=================================================================
% Full title of the paper (Capitalized)
\Title{\textbf{Название: {\Large Название вашей рукописи должно быть кратким, конкретным и релевантным. Пожалуйста, не включайте сокращенные или краткие формы названия, такие как заголовок или заголовок. Они будут удалены нашей редакцией}}}

% i-edu.uz internal command: Title for citation in the left column
\TitleCitation{КАРДИОЛОГИЯ УЗБЕКИСТАНА}


% Author Orchid ID: enter ID or remove command
\newcommand{\orcidauthorA}{0000-0000-0000-0000}
\newcommand{\orcidauthorS}{0000-0000-0000-0000}
\newcommand{\orcidauthorD}{0000-0000-0000-0000}
\newcommand{\orcidauthorG}{0000-0000-0000-0000}  % Add \orcidA{} behind the author's name
%\newcommand{\orcidauthorB}{0000-0000-0000-000X} % Add \orcidB{} behind the author's name

% Authors, for the paper (add full first names)
\Author{Автор (И.О. Фамилия) $^{1}$ \orcidA{}, Автор (И.О. Фамилия) $^{1}$\orcidS{}, Автор (И.О. Фамилия) $^{1}$* \orcidD{}, Автор (И.О. Фамилия) $^{1}$\orcidG{}}

%\longauthorlist{yes}

% i-edu.uz internal command: Authors, for metadata in PDF
\AuthorNames{Автор (И.О. Фамилия), Автор (И.О. Фамилия), Автор (И.О. Фамилия), Автор (И.О. Фамилия)}

% i-edu.uz internal command: Authors, for citation in the left column
\AuthorCitation{Автор (И.О. Фамилия), Автор (И.О. Фамилия), Автор (И.О. Фамилия),Автор (И.О. Фамилия)}
% If this is a Chicago style journal: Lastname, Firstname, Firstname Lastname, and Firstname Lastname.

% Affiliations / Addresses (Add [1] after \address if there is only one affiliation.)
\address{%
$^{1}$ \quad Названия отдела, Названия организации, Город, индекс, Страна\\

$^{}$ \quad e-mail@e-mail.com (И.Ф.), e-mail@e-mail.com (И.Ф.), e-mail@e-mail.com (И.Ф.)\\
}

% Contact information of the corresponding author
\corres{Correspondence: e-mail@e-mail.com; Tel.: +000 00 0000000 (И.Ф.)}

%\simplesumm{} % Simple summary

%\conference{} % An extended version of a conference paper

%\Abstract (Do not insert blank lines, i.e. \\) 

\abstract {\newlineАннотация должна быть в общей сложности около 200 слов максимум. Для научных статей аннотация должна предоставить краткий обзор работы. Мы настоятельно рекомендуем авторам использовать стиль структурированных аннотаций, но без выделенных заголовков.\\
\textbf{Цель.} Цель в научной статье должна четко формулировать основную задачу исследования, обозначая, что именно авторы стремятся достичь или исследовать. Она должна быть краткой, ясной и отражать основное направление работы
\newline\textbf{Материалы и методы.} Кратко опишите основные методы или подходы, которые были применены в исследовании.
\newline\textbf{Результаты.} Кратко изложите основные результаты статьи.
\newline\textbf{Заключение.} 
Укажите основные выводы или интерпретации. Аннотация должна объективно отражать содержание статьи, не содержать результатов, которые не представлены и не обоснованы в основном тексте, и не преувеличивать основные выводы.
}

% Keywords
\keyword{После аннотации необходимо добавить от трех до десяти соответствующих ключевых слов. Мы рекомендуем, чтобы ключевые слова были специфичны для статьи, но при этом достаточно распространены в рамках предметной дисциплины.} \\

% The fields PACS, MSC, and JEL may be left empty or commented out if not applicable
%\PACS{J0101}
%\MSC{}
%\JEL{}

%%%%%%%%%%%%%%%%%%%%%%%%%%%%%%%%%%%%%%%%%%
% Only for the journal Diversity
%\LSID{\url{http://}}

%%%%%%%%%%%%%%%%%%%%%%%%%%%%%%%%%%%%%%%%%%
% Only for the journal Applied Sciences
%\featuredapplication{Authors are encouraged to provide a concise description of the specific application or a potential application of the work. This section is not mandatory.}
%%%%%%%%%%%%%%%%%%%%%%%%%%%%%%%%%%%%%%%%%%

%%%%%%%%%%%%%%%%%%%%%%%%%%%%%%%%%%%%%%%%%%
% Only for the journal Data
%\dataset{DOI number or link to the deposited data set if the data set is published separately. If the data set shall be published as a supplement to this paper, this field will be filled by the journal editors. In this case, please submit the data set as a supplement.}
%\datasetlicense{License under which the data set is made available (CC0, CC-BY, CC-BY-SA, CC-BY-NC, etc.)}

%%%%%%%%%%%%%%%%%%%%%%%%%%%%%%%%%%%%%%%%%%
% Only for the journal Toxins
%\keycontribution{The breakthroughs or highlights of the manuscript. Authors can write one or two sentences to describe the most important part of the paper.}

%%%%%%%%%%%%%%%%%%%%%%%%%%%%%%%%%%%%%%%%%%
% Only for the journal Encyclopedia
%\encyclopediadef{For entry manuscripts only: please provide a brief overview of the entry title instead of an abstract.}

%%%%%%%%%%%%%%%%%%%%%%%%%%%%%%%%%%%%%%%%%%
% Only for the journal Advances in Respiratory Medicine
%\addhighlights{yes}
%\renewcommand{\addhighlights}{%

%\noindent This is an obligatory section in “Advances in Respiratory Medicine”, whose goal is to increase the discoverability and readability of the article via search engines and other scholars. Highlights should not be a copy of the abstract, but a simple text allowing the reader to quickly and simplified find out what the article is about and what can be cited from it. Each of these parts should be devoted up to 2~bullet points.\vspace{3pt}\\
%\textbf{What are the main findings?}
% \begin{itemize}[labelsep=2.5mm,topsep=-3pt]
% \item First bullet.
% \item Second bullet.
% \end{itemize}\vspace{3pt}
%\textbf{What is the implication of the main finding?}
% \begin{itemize}[labelsep=2.5mm,topsep=-3pt]
% \item First bullet.
% \item Second bullet.
% \end{itemize}
%}

%%%%%%%%%%%%%%%%%%%%%%%%%%%%%%%%%%%%%%%%%%
\begin{document}
%%%%%%%%%%%%%%%%%%%%%%%%%%%%%%%%%%%%%%%%%%
\begin{sloppypar}

\section*{\Large \textbf{Title: { \large The title of your manuscript should be concise, specific and relevant. Please do not include abbreviated or short forms of the title, such as a running title or head. These will be removed by our Editorial Office.}}}\\

\begingroup

\raggedright 

\textbf{Name I.Surname}  $^{1}$ \orcidA{}, \textbf{Name I.Surname}  $^{1}$\orcidS{}, \textbf{Name I.Surname}  $^{1}$* \orcidD{}, \textbf{Name I.Surname}  $^{1}$\orcidG{}\\
\endgroup\\ 

%%%%%%%%%%%%%%%%%%%%%%%%%%%%%%%%%%%%%%%%%%%%%%%%%%%%%%

\noindent {\scriptsize{\\ \noindent$^{1}$ \quad Department Name, Organization Name, City, ZIP Code, Country

\noindent$^{}$ \quad e-mail@e-mail.com (F.L.,), e-mail@e-mail.com (F.L.,), e-mail@e-mail.com (F.L.,),e-mail@e-mail.com (F.L.,)}}\\

\noindent\textbf{Abstract:}\\A single paragraph of about 200 words maximum. For research articles, abstracts should give a pertinent overview of the work. We strongly encourage authors to use the following style of structured abstracts, but without headings:
\\
\textbf{Background.} place the question addressed in a broad context and highlight the purpose of the study;

\noindent\textbf{Materials and methods.} describe briefly the main methods or treatments applied; 

\noindent\textbf{Results.}summarize the article's main findings;

\noindent\textbf{Conclusion.} indicate the main conclusions or interpretations. The abstract should be an objective representation of the article, it must not contain results which are not presented and substantiated in the main text and should not exaggerate the main conclusions.\\

\noindent\textbf{ Key words:} {keyword 1; keyword 2; keyword 3 (List three to ten pertinent keywords specific to the article; yet reasonably common within the subject discipline.)\\

%%%%%%%%%%%%%%%%%%%%%%%%%%%%%%%%%%%%%%%%%%
\setcounter{section}{-0} %% Remove this when starting to work on the template.

\noindent\textbf{Введение}

Введение должно кратко представить исследование в широком контексте и подчеркнуть его значимость. Необходимо четко определить цель работы и её важность. Текущая ситуация в области исследования должна быть тщательно рассмотрена с упоминанием ключевых публикаций. При необходимости следует выделить спорные и расходящиеся гипотезы. В завершение, кратко укажите основную цель работы и подчеркните главные выводы. По возможности постарайтесь сделать введение понятным для ученых за пределами вашей конкретной области исследования. Ссылка на статью журнала \cite{ref-journal}. Ссылка на книгу \cite{ref-book1,ref-book2} или другие виды ссылок \cite{ref-unpublish,ref-communication,ref-proceeding}.



%\endnote{This is an endnote.} % To use endnotes, please un-comment \printendnotes below (before References). Only journal Laws uses \footnote.

% The order of the section titles is different for some journals. Please refer to the "Instructions for Authors” on the journal homepage.

\noindent\textbf{Материалы и методы}

 Должен быть описан достаточно подробно, чтобы другие могли воспроизвести и опираться на опубликованные результаты. Обратите внимание, что публикация вашей рукописи подразумевает обязательство предоставить всем читателям доступ ко всем материалам, данным, программному коду и протоколам, связанным с публикацией. Пожалуйста, укажите на этапе подачи любые ограничения на доступность материалов или информации. Новые методы и протоколы должны быть описаны подробно, в то время как хорошо установленные методы могут быть описаны кратко с соответствующими ссылками.

Рукописи исследований, сообщающие о крупных наборах данных, размещенных в общедоступных базах данных, должны указывать, где данные были размещены, и предоставлять соответствующие номера доступа. Если номера доступа еще не были получены на момент подачи, укажите, что они будут предоставлены в процессе рецензирования. Номера должны быть предоставлены до публикации.

Интервенционные исследования, включающие животных или людей, а также другие исследования, требующие этического одобрения, должны содержать информацию об органе, предоставившем одобрение, и указать соответствующий этический код одобрения

\noindent\textbf{Результаты}

Этот раздел может быть разделен на подразделы. Он должен содержать краткое и точноописание экспериментальных результатов, их интерпретацию, а также выводы, которые можно сделать на основе эксперимента.
\subsection{Подраздел}
\subsubsection{Подподраздел}

Маркированные списки выглядят так:
\begin{itemize}
\item	Пункт первый;
\item	Пункт второй;
\item	Пункт третий.
\end{itemize}

Нумерованные списки можно добавить следующим образом::
\begin{enumerate}
\item	Пункт первый;
\item	Пункт второй;
\item	Пункт третий.
\end{enumerate}

Текст продолжается здесь. 

\subsection{Ресунки, таблицы и схемы}

Все фигуры и таблицы должны быть процитированы в основном тексте как Рисунок~\ref{fig1}, Таблица~\ref{tab1} и т.д.

\begin{figure}[H]
\centering
\includegraphics[width=10.5 cm]{issledovania.jpg}
\caption{\centeringЭто рисунок.\\ Схемы следует оформлять по аналогичным правилам. Если в рисунке несколько панелей, их следует перечислять так: (\textbf{a}) Описание содержимого первой панели. (\textbf{b}) Описание содержимого второй панели. Рисунки следует размещать в основном тексте рядом с первым упоминанием. Подпись должна быть выровнена по центру и размещена на одной строке.\\
\textbf{Fig.1.} This is a drawing.\label{fig1}}
\end{figure}  
\\

\unskip
\begin{table}[H] 
\caption{Это подпись к таблице. Таблицы следует размещать в основном тексте рядом с первым упоминанием.\label{tab1}\\
\textbf{Table 1.}This is the signature to the table. Tables should be placed in the main text next to the first mention.\label{tab1}}
\newcolumntype{C}{>{\centering\arraybackslash}X}
\begin{tabularx}{\textwidth}{CCC}
\toprule
\textbf{Title 1}	& \textbf{Title 2}	& \textbf{Title 3}\\
\midrule
Entry 1		& Data			& Data\\
Entry 2		& Data			& Data \textsuperscript{1}\\
\bottomrule
\end{tabularx}
\noindent{\footnotesize{\textsuperscript{1} Таблицы могут иметь нижний колонтитул.}}
\end{table}

Текст продолжается здесь (Рисунок~\ref{fig2} и Таблица~\ref{tab2}).

% Example of a figure that spans the whole page width. The same concept works for tables, too.
\begin{figure}[H]
\begin{adjustwidth}{-\extralength}{0cm}
\centering
\includegraphics[width=15.5cm]{issledovania.jpg}
\end{adjustwidth}
\caption{Это широкий рисунок.\label{fig2}\\
\textbf{Fig.2.}This is a wide drawing.\label{fig2}}
\end{figure}  

\begin{table}[H]
\caption{Это широкая таблица.\label{tab2}\\
\textbf{Table 2}This is a wide table.\label{tab2}}
	\begin{adjustwidth}{-\extralength}{0cm}
		\newcolumntype{C}{>{\centering\arraybackslash}X}
		\begin{tabularx}{\fulllength}{CCCC}
			\toprule
			\textbf{Title 1}	& \textbf{Title 2}	& \textbf{Title 3}     & \textbf{Title 4}\\
			\midrule
\multirow[m]{3}{*}{Entry 1 *}	& Data			& Data			& Data\\
			  	                   & Data			& Data			& Data\\
			             	      & Data			& Data			& Data\\
                   \midrule
\multirow[m]{3}{*}{Entry 2}    & Data			& Data			& Data\\
			  	                  & Data			& Data			& Data\\
			             	     & Data			& Data			& Data\\
                   \midrule
\multirow[m]{3}{*}{Entry 3}    & Data			& Data			& Data\\
			  	                 & Data			& Data			& Data\\
			             	    & Data			& Data			& Data\\
                  \midrule
\multirow[m]{3}{*}{Entry 4}   & Data			& Data			& Data\\
			  	                 & Data			& Data			& Data\\
			             	    & Data			& Data			& Data\\
			\bottomrule
		\end{tabularx}
	\end{adjustwidth}
	\noindent{\footnotesize{* Таблицы могут иметь нижний колонтитул.}}
\end{table}

%\begin{listing}[H]
%\caption{Title of the listing}
%\rule{\columnwidth}{1pt}
%\raggedright Text of the listing. In font size footnotesize, small, or normalsize. Preferred format: left aligned and single spaced. Preferred border format: top border line and bottom border line.
%\rule{\columnwidth}{1pt}
%\end{listing}

Текст.

Текст.

\subsection{Форматирование математических компонентов}

Это пример уравнения 1:
\begin{linenomath}
\begin{equation}
a = 1,
\end{equation}
\end{linenomath}
Текст после уравнения не обязательно должен начинаться с нового абзаца. Пожалуйста, ставьте пунктуацию в уравнениях как в обычном тексте.
%% If the documentclass option "submit" is chosen, please insert a blank line before and after any math environment (equation and eqnarray environments). This ensures correct linenumbering. The blank line should be removed when the documentclass option is changed to "accept" because the text following an equation should not be a new paragraph.

Это пример уравнения 2:
\begin{adjustwidth}{-\extralength}{0cm}
\begin{equation}
a = b + c + d + e + f + g + h + i + j + k + l + m + n + o + p + q + r + s + t + u + v + w + x + y + z
\end{equation}
\end{adjustwidth}

% Example of a page in landscape format (with table and table footnote).
%\startlandscape
%\begin{table}[H] %% Table in wide page
%\caption{This is a very wide table.\label{tab3}}
%	\begin{tabularx}{\textwidth}{CCCC}
%		\toprule
%		\textbf{Title 1}	& \textbf{Title 2}	& \textbf{Title 3}	& \textbf{Title 4}\\
%		\midrule
%		Entry 1		& Data			& Data			& This cell has some longer content that runs over two lines.\\
%		Entry 2		& Data			& Data			& Data\textsuperscript{1}\\
%		\bottomrule
%	\end{tabularx}
%	\begin{adjustwidth}{+\extralength}{0cm}
%		\noindent\footnotesize{\textsuperscript{1} This is a table footnote.}
%	\end{adjustwidth}
%\end{table}
%\finishlandscape


Пожалуйста, ставьте пунктуацию в уравнениях как в обычном тексте. Среды типа теорем (включая предложения, леммы, следствия и т.д.) можно форматировать следующим образом:
%% Example of a theorem:
\begin{Theorem}
\textbf{Пример текста теоремы.}
\end{Theorem}

Текст продолжается здесь. Доказательства должны быть отформатированы следующим образом:

%% Example of a proof:

\begin{Theorem}
\textbf{Доказательство теоремы 1}

Текст доказательства. Обратите внимание, что фраза «теорема 1» является необязательной, если очевидно, о какой теореме идет речь.
\end{Theorem}
Текст продолжается здесь.\\

%%%%%%%%%%%%%%%%%%%%%%%%%%%%%%%%%%%%%%%%%%
\noindent\textbf{Обсуждение:}

Авторы должны обсудить результаты и их интерпретацию в контексте предыдущих исследований и рабочей гипотезы. Находки и их последствия следует обсуждать в самом широком контексте. Также можно выделить направления для будущих исследований.

%%%%%%%%%%%%%%%%%%%%%%%%%%%%%%%%%%%%%%%%%%
\vspace{6pt} 

%%%%%%%%%%%%%%%%%%%%%%%%%%%%%%%%%%%%%%%%%%
%% optional
%\supplementary{The following supporting information can be downloaded at:  \linksupplementary{s1}, Figure S1: title; Table S1: title; Video S1: title.}

% Only for journal Methods and Protocols:
% If you wish to submit a video article, please do so with any other supplementary material.
% \supplementary{The following supporting information can be downloaded at: \linksupplementary{s1}, Figure S1: title; Table S1: title; Video S1: title. A supporting video article is available at doi: link.}

% Only for journal Hardware:
% If you wish to submit a video article, please do so with any other supplementary material.
% \supplementary{The following supporting information can be downloaded at: \linksupplementary{s1}, Figure S1: title; Table S1: title; Video S1: title.\vspace{6pt}\\
%\begin{tabularx}{\textwidth}{lll}
%\toprule
%\textbf{Name} & \textbf{Type} & \textbf{Description} \\
%\midrule
%S1 & Python script (.py) & Script of python source code used in XX \\
%S2 & Text (.txt) & Script of modelling code used to make Figure X \\
%S3 & Text (.txt) & Raw data from experiment X \\
%S4 & Video (.mp4) & Video demonstrating the hardware in use \\
%... & ... & ... \\
%\bottomrule
%\end{tabularx}
%}

%%%%%%%%%%%%%%%%%%%%%%%%%%%%%%%%%%%%%%%%%%
%%%%%%%%%%%%%%%%%%%%%%%%%%%%%%%%%%%%%%%%%%

\noindent\textbf{Заключение}

Этот раздел не является обязательным, но может быть добавлен к рукописи, если обсуждение является особенно длинным или сложным.

%%%%%%%%%%%%%%%%%%%%%%%%%%%%%%%%%%%%%%%%%%
\vspace{6pt} 

%%%%%%%%%%%%%%%%%%%%%%%%%%%%%%%%%%%%%%%%%%
%% optional
%\supplementary{The following supporting information can be downloaded at:  \linksupplementary{s1}, Figure S1: title; Table S1: title; Video S1: title.}

% Only for journal Methods and Protocols:
% If you wish to submit a video article, please do so with any other supplementary material.
% \supplementary{The following supporting information can be downloaded at: \linksupplementary{s1}, Figure S1: title; Table S1: title; Video S1: title. A supporting video article is available at doi: link.}

% Only for journal Hardware:
% If you wish to submit a video article, please do so with any other supplementary material.
% \supplementary{The following supporting information can be downloaded at: \linksupplementary{s1}, Figure S1: title; Table S1: title; Video S1: title.\vspace{6pt}\\
%\begin{tabularx}{\textwidth}{lll}
%\toprule
%\textbf{Name} & \textbf{Type} & \textbf{Description} \\
%\midrule
%S1 & Python script (.py) & Script of python source code used in XX \\
%S2 & Text (.txt) & Script of modelling code used to make Figure X \\
%S3 & Text (.txt) & Raw data from experiment X \\
%S4 & Video (.mp4) & Video demonstrating the hardware in use \\
%... & ... & ... \\
%\bottomrule
%\end{tabularx}
%}

%%%%%%%%%%%%%%%%%%%%%%%%%%%%%%%%%%%%%%%%%%
\authorcontributionsr

Для исследовательских статей с несколькими авторами необходимо предоставить короткий абзац, уточняющий их индивидуальные вклады. Следует использовать следующие формулировки: "Концептуализация, X.X. и Y.Y.; методология, X.X.; программное обеспечение, X.X.; валидация, X.X., Y.Y. и Z.Z.; формальный анализ, X.X.; исследование, X.X.; ресурсы, X.X.; кураторство данных, X.X.; написание оригинального текста, X.X.; написание и редактирование, X.X.; визуализация, X.X.; руководство, X.X.; администрирование проекта, X.X.; привлечение финансирования, Y.Y. Все авторы ознакомлены с опубликованной версией рукописи и согласны с ней." Пожалуйста, обратитесь к \href{http://img.mdpi.org/data/contributor-role-instruction.pdf}{таксономии CRediT} для объяснения терминов. Авторство должно быть ограничено теми, кто внес существенный вклад в представленную работу.

\authorcontributions

For research articles with several authors, a short paragraph specifying their individual contributions must be provided. The following statements should be used ``Conceptualization, X.X. and Y.Y.; methodology, X.X.; software, X.X.; validation, X.X., Y.Y. and Z.Z.; formal analysis, X.X.; investigation, X.X.; resources, X.X.; data curation, X.X.; writing---original draft preparation, X.X.; writing---review and editing, X.X.; visualization, X.X.; supervision, X.X.; project administration, X.X.; funding acquisition, Y.Y. All authors have read and agreed to the published version of the manuscript.'', please turn to the  \href{http://img.mdpi.org/data/contributor-role-instruction.pdf}{CRediT taxonomy} for the term explanation. Authorship must be limited to those who have contributed substantially to the work~reported.

\fundingr

{Пожалуйста, добавьте: «Это исследование не получало внешнего финансирования» или «Это исследование было профинансировано грантом NAME OF FUNDER, номер XXX». Также добавьте «APC был профинансирован XXX». Тщательно проверьте, чтобы предоставленные данные были точными, и используйте стандартное написание названий фондов на сайте \url{https://search.crossref.org/funding}; любые ошибки могут повлиять на ваше будущее финансирование}

\funding

{Please add: ``This research received no external funding'' or ``This research was funded by NAME OF FUNDER grant number XXX.'' and  and ``The APC was funded by XXX''. Check carefully that the details given are accurate and use the standard spelling of funding agency names at \url{https://search.crossref.org/funding}, any errors may affect your future funding.}}

\institutionalreviewr

{В этом разделе следует добавить заявление об одобрении институционального обзорного комитета и номер одобрения, если это имеет отношение к вашему исследованию. Вы можете выбрать исключить это заявление, если исследование не требовало этического одобрения. Обратите внимание, что редакционная коллегия может запросить у вас дополнительную информацию. Пожалуйста, добавьте: «Исследование проводилось в соответствии с Декларацией Хельсинки и было одобрено Институциональным обзорным комитетом (или Этическим комитетом) ИМЯ ИНСТИТУТА (код протокола XXX и дата одобрения).» для исследований, включающих людей. ИЛИ «Протокол исследования на животных был одобрен Институциональным обзорным комитетом (или Этическим комитетом) ИМЯ ИНСТИТУТА (код протокола XXX и дата одобрения).» для исследований, включающих животных. ИЛИ «Этический обзор и одобрение были отменены для этого исследования из-за ПРИЧИНА (пожалуйста, предоставьте подробное обоснование).» ИЛИ «Не применимо» для исследований, не включающих людей или животных.}

\institutionalreview

{In this section, you should add the Institutional Review Board Statement and approval number, if relevant to your study. You might choose to exclude this statement if the study did not require ethical approval. Please note that the Editorial Office might ask you for further information. Please add “The study was conducted in accordance with the Declaration of Helsinki, and approved by the Institutional Review Board (or Ethics Committee) of NAME OF INSTITUTE (protocol code XXX and date of approval).” for studies involving humans. OR “The animal study protocol was approved by the Institutional Review Board (or Ethics Committee) of NAME OF INSTITUTE (protocol code XXX and date of approval).” for studies involving animals. OR “Ethical review and approval were waived for this study due to REASON (please provide a detailed justification).” OR “Not applicable” for studies not involving humans or animals}

\informedconsentr

{Любая научная статья, описывающая исследование, в котором участвуют люди, должна содержать следующее заявление. Пожалуйста, добавьте: «Информированное согласие было получено от всех участников исследования.» ИЛИ «Согласие пациентов было отменено из-за ПРИЧИНА (пожалуйста, предоставьте подробное обоснование).» ИЛИ «Не применимо» для исследований, не включающих людей. Вы также можете выбрать исключение этого заявления, если исследование не включало людей. Письменное информированное согласие на публикацию должно быть получено от участвующих пациентов, которые могут быть идентифицированы (включая самих пациентов). Пожалуйста, укажите «Письменное информированное согласие было получено от пациента(ов) для публикации данной статьи», если это применимо.}

\informedconsent

{Any research article describing a study involving humans should contain this statement. Please add ``Informed consent was obtained from all subjects involved in the study.'' OR ``Patient consent was waived due to REASON (please provide a detailed justification).'' OR ``Not applicable'' for studies not involving humans. You might also choose to exclude this statement if the study did not involve humans. 
Written informed consent for i-edu must be obtained from participating patients who can be identified (including by the patients themselves). Please state ``Written informed consent has been obtained from the patient(s) to publish this paper'' if applicable.}

\dataavailabilityr

{Мы призываем всех авторов статей, опубликованных в журнале, делиться своими исследовательскими данными. В этом разделе, пожалуйста, укажите, где можно найти данные, поддерживающие представленные результаты, включая ссылки на общедоступные архивированные наборы данных, которые были проанализированы или сгенерированы в ходе исследования. Если новые данные не были созданы или данные недоступны из-за конфиденциальности или этических ограничений, необходимо предоставить соответствующее заявление.}

\dataavailability

{We encourage all authors of articles published in the journal to share their research data. In this section, please specify where the data supporting the reported results can be found, including links to publicly archived datasets analyzed or generated during the study. If no new data were created or if data is unavailable due to privacy or ethical restrictions, a statement is still required.}

% Only for journal Nursing Reports
%\publicinvolvement{Please describe how the public (patients, consumers, carers) were involved in the research. Consider reporting against the GRIPP2 (Guidance for Reporting Involvement of Patients and the Public) checklist. If the public were not involved in any aspect of the research add: ``No public involvement in any aspect of this research''.}

% Only for journal Nursing Reports
%\guidelinesstandards{Please add a statement indicating which reporting guideline was used when drafting the report. For example, ``This manuscript was drafted against the XXX (the full name of reporting guidelines and citation) for XXX (type of research) research''. A complete list of reporting guidelines can be accessed via the equator network: \url{https://www.equator-network.org/}.}

% Only for journal Nursing Reports
%\useofartificialintelligence{Please describe in detail any and all uses of artificial intelligence (AI) or AI-assisted tools used in the preparation of the manuscript. This may include, but is not limited to, language translation, language editing and grammar, or generating text. Alternatively, please state that “AI or AI-assisted tools were not used in drafting any aspect of this manuscript”.}

%\acknowledgments{Не п.}
\acknowledgmentsr

{In this section you can acknowledge any support given which is not covered by the author contribution or funding sections. This may include administrative and technical support, or donations in kind (e.g., materials used for experiments).}

\acknowledgments

{In this section you can acknowledge any support given which is not covered by the author contribution or funding sections. This may include administrative and technical support, or donations in kind (e.g., materials used for experiments).}

\conflictsofinterestr

{"Укажите конфликты интересов или заявите: «Авторы заявляют об отсутствии конфликта интересов». Авторы должны выявить и заявить о любых личных обстоятельствах или интересах, которые могут восприниматься как неуместно влияющие на представление или интерпретацию результатов исследования. Любая роль спонсоров в разработке исследования, в сборе, анализе или интерпретации данных, в написании рукописи или в принятии решения о публикации результатов должна быть указана в этом разделе. Если роли спонсоров не было, укажите: «Спонсоры не участвовали в разработке исследования; в сборе, анализе или интерпретации данных; в написании рукописи или в принятии решения о публикации результатов».} 

\conflictsofinterest

{Declare conflicts of interest or state ``The authors declare no conflicts of interest.'' Authors must identify and declare any personal circumstances or interest that may be perceived as inappropriately influencing the representation or interpretation of reported research results. Any role of the funders in the design of the study; in the collection, analyses or interpretation of data; in the writing of the manuscript; or in the decision to publish the results must be declared in this section. If there is no role, please state ``The funders had no role in the design of the study; in the collection, analyses, or interpretation of data; in the writing of the manuscript; or in the decision to publish the results''.}
%%%%%%%%%%%%%%%%%%%%%%%%%%%%%%%%%%%%%%%%%%

\abbreviations{Сокращения}
{В данной рукописи используются следующие сокращения:\\

\noindent 
\begin{tabular}{@{}ll}
WoS  &  web of science\\
DOAJ & Directory of open access journals\\
TLA & Three letter acronym\\
LD & Linear dichroism
\end{tabular}
}
%%%%%%%%%%%%%%%%%%%%%%%%%%%%%%%%%%%%%%%%%%
%% Optional

%% Only for journal Encyclopedia
%\entrylink{The Link to this entry published on the encyclopedia platform.}

%\abbreviations{Abbreviations}{
%В данной рукописи использованы следующие сокращения:\\
%}


%%%%%%%%%%%%%%%%%%%%%%%%%%%%%%%%%%%%%%%%%%

\reftitle{\textbf{Литература}}

% Please provide either the correct journal abbreviation (e.g. according to the “List of Title Word Abbreviations” http://www.issn.org/services/online-services/access-to-the-ltwa/) or the full name of the journal.
% Citations and References in Supplementary files are permitted provided that they also appear in the reference list here. 

%=====================================
% References, variant A: external bibliography
%=====================================
%\bibliography{your_external_BibTeX_file}

%=====================================
% References, variant B: internal bibliography
%=====================================
\begin{thebibliography}{999}
% Reference 1
\bibitem[Author1(year)]{ref-journal}
Author~1, T. The title of the cited article. {\em Journal Abbreviation} {\bf 2008}, {\em 10}, 142--149.
% Reference 2
\bibitem[Author2(year)]{ref-book1}
Author~2, L. The title of the cited contribution. In {\em The Book Title}; Editor 1, F., Editor 2, A., Eds.; Publishing House: City, Country, 2007; pp. 32--58.
% Reference 3
\bibitem[Author3(year)]{ref-book2}
Author 1, A.; Author 2, B. \textit{Book Title}, 3rd ed.; Publisher: Publisher Location, Country, 2008; pp. 154--196.
% Reference 4
\bibitem[Author4(year)]{ref-unpublish}
Author 1, A.B.; Author 2, C. Title of Unpublished Work. \textit{Abbreviated Journal Name} year, \textit{phrase indicating stage of i-edu (submitted; accepted; in press)}.
% Reference 5
\bibitem[Author5(year)]{ref-communication}
Author 1, A.B. (University, City, State, Country); Author 2, C. (Institute, City, State, Country). Personal communication, 2012.
% Reference 6
\bibitem[Author6(year)]{ref-proceeding}
Author 1, A.B.; Author 2, C.D.; Author 3, E.F. Title of presentation. In Proceedings of the Name of the Conference, Location of Conference, Country, Date of Conference (Day Month Year); Abstract Number (optional), Pagination (optional).
% Reference 7
\bibitem[Author7(year)]{ref-thesis}
Author 1, A.B. Title of Thesis. Level of Thesis, Degree-Granting University, Location of University, Date of Completion.
% Reference 8
\bibitem[Author8(year)]{ref-url}
Title of Site. Available online: URL (accessed on Day Month Year).\\
\end{thebibliography}


% If authors have biography, please use the format below
%\section*{Short Biography of Authors}
%\bio
%{\raisebox{-0.35cm}{\includegraphics[width=3.5cm,height=5.3cm,clip,keepaspectratio]{Definitions/author1.pdf}}}
%{\textbf{Firstname Lastname} Biography of first author}
%
%\bio
%{\raisebox{-0.35cm}{\includegraphics[width=3.5cm,height=5.3cm,clip,keepaspectratio]{Definitions/author2.jpg}}}
%{\textbf{Firstname Lastname} Biography of second author}

% For the i-edu.uz journals use author-date citation, please follow the formatting guidelines on http://www.i-edu.uz/authors/references
% To cite two works by the same author: \citeauthor{ref-journal-1a} (\citeyear{ref-journal-1a}, \citeyear{ref-journal-1b}). This produces: Whittaker (1967, 1975)
% To cite two works by the same author with specific pages: \citeauthor{ref-journal-3a} (\citeyear{ref-journal-3a}, p. 328; \citeyear{ref-journal-3b}, p.475). This produces: Wong (1999, p. 328; 2000, p. 475)

%%%%%%%%%%%%%%%%%%%%%%%%%%%%%%%%%%%%%%%%%%
%% for journal Sci
%\reviewreports{\\
%Reviewer 1 comments and authors’ response\\
%Reviewer 2 comments and authors’ response\\
%Reviewer 3 comments and authors’ response
%}
%%%%%%%%%%%%%%%%%%%%%%%%%%%%%%%%%%%%%%%%%%
%\PublishersNote{}
\\
\noindent\textbf{Отказ от ответственности/Примечание издателя:} Заявления, мнения и данные, содержащиеся во всех публикациях, принадлежат исключительно отдельным лицам.Авторы и участники, а Журнал и редакторы. Журнал и редакторы не несут ответственности за любой ущерб, нанесенныйлюдей или имущество, возникшее в результате любых идей, методов, инструкций или продуктов, упомянутых в контенте.\\

\noindent\textbf{Disclaimer of liability/Publisher's Note:} The statements, opinions and data contained in all publications belong exclusively to individuals.The authors and participants, and the Journal and the editors. The journal and the editors are not responsible for any damage caused to people or property resulting from any ideas, methods, instructions or products mentioned in the content.

\end{sloppypar}
\end{document}


 